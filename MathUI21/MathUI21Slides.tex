\newif\ifbeamer
\beamertrue
\def\JHD#1{\ifbeamer\end{frame}\begin{frame}\frametitle{#1}\else\section{#1}\fi}
\def\red#1{{\color{red}#1}}
\def\redtt{\color{red}\tt}
\def\bluett{\color{blue}\tt}
\font\manual=manfnt
\def\dbend{{\manual\char127}}
\def\eqq{{\buildrel?\over=}}
%\def\B{\red{$\cal B$}}
\def\lcm{\mathop{\rm lcm}}
\def\lm{\mathop{\rm lm}}
\def\I{{\cal I}}
\def\Z{{\bf Z}}
\def\Q{{\mathbb Q}}
\def\C{{\mathbb C}}
\def\N{{\bf N}}
\def\R{{\bf R}}
\def\CL{\mathop{\rm CL}}
\def\IDA{{}_{\rm DA}\int}
\def\DDA{D_{\rm DA}}
\def\arctanDA{\arctan_{\rm DA}}
\def\logDA{\log_{\rm DA}}
\def\cDA{constant${}_{\rm DA}$}
\def\fracDDA#1#2{\frac{{\rm d}#1}{{\rm d}_{\rm DA}#2}}
\def\Ied{{}_{\epsilon\delta}\int}
\def\Ded{D_{\epsilon\delta}}
\def\arctaned{\arctan_{\epsilon\delta}}
\def\loged{\log_{\epsilon\delta}}
\def\fracDed#1#2{\frac{{\rm d}#1}{{\rm d}_{\epsilon\delta}#2}}
\def\Book{\red{$\cal C$}}
\def\J{\red{$\cal J$}}
\def\boxit#1{\lower6pt\vbox{\hrule\hbox{\vrule\kern3pt\vbox{\kern3pt\hbox{#1}\kern3pt}\kern3pt\vrule}\hrule}}
\ifbeamer
\documentclass[Aspectratio=169,notes=hide]{beamer}   % Without notes
%\documentclass[Aspectratio=169,handout]{beamer}   % handout mode
\usetheme{Warsaw}

\setbeamertemplate{navigation symbols}{}
\else
\def\pause{}
\documentclass{article}
\usepackage[mathjax]{lwarp}
\usepackage{lmodern}
\fi
\usepackage[final]{pdfpages}


\usepackage{verbatim}
\usepackage{url}
\usepackage{color}
\usepackage[version=3]{mhchem}
\bibliographystyle{alphaurl}
\begin{document}
\title{ATLAS: Interactive and Educational Linear Algebra System Containing Non-Standard Methods}
\author[PaiDavenport]{
\underline{Akhilesh Pai}\inst{1}%\thanks{}
\and
    \underline{James H. Davenport}\inst{1}%\thanks{Support of EPSRC (Grant EP/T015713/1) is gratefully acknowledged.}
}
\def\email#1{{\tt #1}}
% Institutes for affiliations are also joined by \and,
\institute{
   University of Bath,
   Bath BA2 7AY, United Kingdom\\
   \email{\{abp34;J.H.Davenport\}@bath.ac.uk}
 }


%\author{James Davenport}%\thanks{Partially supported by EU H2020 project SC$^2$ (712689), and wholly supported by DEWCAD colleagues}}
%(Thanks to RJB for the improved title)}
\date{30 July 2021 at MathUI/CICM 2021}
\expandafter\def\expandafter\insertshorttitle\expandafter{%
  \insertshorttitle\hfill%
  \insertframenumber\,/\,\inserttotalframenumber}
\frame{
\titlepage
} 
\begin{frame}[fragile]
\frametitle{Plan of Talk}
\begin{enumerate}%[<+->]
\item Introduction
\item Prior Work
\item Current State
\item Demonstration
\item Q\&A
\end{enumerate}
\end{frame}
\begin{frame}
\frametitle{Introduction}%\pause
\begin{itemize}[<+->]
\item Numerous linear algebra teaching tools currently exist such as Maple, MATLAB and CoCalc
\item But these tools primarily focus on the basic methods
\item More advanced methods are absent from these tools
\item Most of these tools also do not provide step-by-step solutions for problems
\item This project aims to fill that gap, focusing specifically on methods like Strassen's fast matrix multiplication and other less common methods
\item ATLAS aims to also provide the ability to compare the step-by-step solutions of methods simultaneously
\item By providing non-standard methods and step-by-step solutions for all methods, users have a greater choice and they can see exactly how problems can be solved using the methods
\end{itemize}
\end{frame}

\begin{frame}%[allowframebreaks]
\frametitle{Prior Work: see \cite{PaiDavenport2021a}}%\pause
No work on on-standard methods.\pause
\begin{itemize}[<+->]
\item[Maple] Maple has a feature called Tutor, allowing users to view the step-by-step solutions to a problem for linear algebra functions, such as calculation of eigenvalues \cite{eigentutor}.
% Tutor may explain the popularity of Maple amongst students, as it can serves as a key tool in their education. However, this feature is also limited to only standard methods. Developing a Tutor-style feature in a linear algebra system, however, combined with non-standard methods could allow different methods to be compared against each other.\newline

\item[MATLAB] 
%MATLAB does not contain a step-by-step solution feature. But users need to learn a new programming language to utilise it. It also does not offer alternative methods, such as Strassen's method. %, limiting itself to only standard methods of matrix multiplication. %MATLAB consequently does not allow methods to be compared with each other. 
\cite{matlab} studied whether MATLAB can %act as a useful tool to 
supplement the learning of university students. Most students liked using MATLAB, with specific preference amongst lower achieving students as they liked getting an answer without working it out, which is not very helpful for their learning. Some students, however, did not like it because it did not show how it got to the answer and they wanted to see how the problem was solved step-by-step, otherwise it wouldn't help them. %Some students also stated that they were ``not good at computers'', highlighting the need for simplicity in a linear algebra system for users who are unfamiliar with computers or programming. %The process of learning a new piece of software is daunting, so it should be made easy.

\item[CoCalc] CoCalc \cite{Stein2018a} is web-based linear algebra system, which makes it more portable than Maple or MATLAB. 
%CoCalc allows collaboration by multiple collaborators simultaneously \cite{collab}, in addition to allowing 
Cocal allows teachers to conduct their lessons entirely on CoCalc through an interactive platform. %Whilst this is an interesting idea, it does little to directly help students to learn independently. Based on the features available, 
CoCalc acts more as an e-learning platform such as Moodle.%, used by many universities. 
%CoCalc also suffers some of the flaws of MATLAB, such that users would need to learn new programming languages to utilise CoCalc.
\end{itemize}
\end{frame}

\begin{frame}%[allowframebreaks]
\frametitle{Pedagogical Example: Strassen--Winograd}\pause
\begin{itemize}[<+->]
\item Very hard to motivate these.
\item But matrix multiplication is bilinear, and these expressions are also bilinear (verification easy).
\item Hence it suffices to verify
$$
\begin{pmatrix}
a&b\cr c&d
\end{pmatrix}\times
\begin{pmatrix}
1&0\cr0&0
\end{pmatrix}
=\begin{pmatrix}
a&0\cr c&0
\end{pmatrix}
$$
(and three analogues)
\item Tedious by hand, but easy with this tool.
\end{itemize}
\end{frame}

\begin{frame}%[allowframebreaks]
\frametitle{Current State (1)}\pause
\begin{itemize}[<+->]
\item ATLAS supports the calculation of determinants, inverses, eigenvectors and eigenvalues, in addition to matrix multiplication and solving systems of linear equations.
\item For calculating determinants, Laplace expansion, Sarrus' method and LU decomposition are supported, allowing users to compare all 3 methods step-by-step simultaneously.
\item This is also true of matrix multiplication, which supports the standard method, Strassen's method and the Laderman method, and calculation of inverses using both the Cramer's rule and the Cayley-Hamilton theorem.
\item Systems of linear equations can be solved by Gaussian elimination, Cramer's rule and Cholesky decomposition, whilst being compared with each other simultaneously
\end{itemize}
\end{frame}

\begin{frame}%[allowframebreaks]
\frametitle{Current State (2)}\pause
\begin{itemize}[<+->]
\item Unit testing completed.
\item Integration and user testing currently in progress.
\item The next stage of development is the implementation of symbolic inputs, as ATLAS is currently limited to only numerical inputs.
\item There is a desire to extend to comparisons of methods that are considered numerically good or bad to understand the effect of different methods on problems with floating point numbers.
\item Also, there is an aspiration to improve the portability of ATLAS by creating a web-based equivalent, similar to CoCalc.
\end{itemize}
\end{frame}

\begin{frame}
\frametitle{Demonstration}%\pause
\begin{itemize}[<+->]
\item[Demo] Now we shall demonstrate some of the functionality of ATLAS
\end{itemize}
\end{frame}

\begin{frame}
\frametitle{Q\&A}%\pause
\begin{description}[<+->]
\item[?]Any questions?
\end{description}
\end{frame}

\begin{frame}[allowframebreaks]
\frametitle{Bibliography}
\bibliography{references}
\end{frame}


\end{document}
